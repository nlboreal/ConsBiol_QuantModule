\documentclass[]{book}
\usepackage{lmodern}
\usepackage{amssymb,amsmath}
\usepackage{ifxetex,ifluatex}
\usepackage{fixltx2e} % provides \textsubscript
\ifnum 0\ifxetex 1\fi\ifluatex 1\fi=0 % if pdftex
  \usepackage[T1]{fontenc}
  \usepackage[utf8]{inputenc}
\else % if luatex or xelatex
  \ifxetex
    \usepackage{mathspec}
  \else
    \usepackage{fontspec}
  \fi
  \defaultfontfeatures{Ligatures=TeX,Scale=MatchLowercase}
\fi
% use upquote if available, for straight quotes in verbatim environments
\IfFileExists{upquote.sty}{\usepackage{upquote}}{}
% use microtype if available
\IfFileExists{microtype.sty}{%
\usepackage{microtype}
\UseMicrotypeSet[protrusion]{basicmath} % disable protrusion for tt fonts
}{}
\usepackage{hyperref}
\hypersetup{unicode=true,
            pdftitle={Conservation Biology - Quantitative Methods Module},
            pdfauthor={Yolanda Wiersma},
            pdfborder={0 0 0},
            breaklinks=true}
\urlstyle{same}  % don't use monospace font for urls
\usepackage{natbib}
\bibliographystyle{apalike}
\usepackage{longtable,booktabs}
\usepackage{graphicx,grffile}
\makeatletter
\def\maxwidth{\ifdim\Gin@nat@width>\linewidth\linewidth\else\Gin@nat@width\fi}
\def\maxheight{\ifdim\Gin@nat@height>\textheight\textheight\else\Gin@nat@height\fi}
\makeatother
% Scale images if necessary, so that they will not overflow the page
% margins by default, and it is still possible to overwrite the defaults
% using explicit options in \includegraphics[width, height, ...]{}
\setkeys{Gin}{width=\maxwidth,height=\maxheight,keepaspectratio}
\IfFileExists{parskip.sty}{%
\usepackage{parskip}
}{% else
\setlength{\parindent}{0pt}
\setlength{\parskip}{6pt plus 2pt minus 1pt}
}
\setlength{\emergencystretch}{3em}  % prevent overfull lines
\providecommand{\tightlist}{%
  \setlength{\itemsep}{0pt}\setlength{\parskip}{0pt}}
\setcounter{secnumdepth}{5}
% Redefines (sub)paragraphs to behave more like sections
\ifx\paragraph\undefined\else
\let\oldparagraph\paragraph
\renewcommand{\paragraph}[1]{\oldparagraph{#1}\mbox{}}
\fi
\ifx\subparagraph\undefined\else
\let\oldsubparagraph\subparagraph
\renewcommand{\subparagraph}[1]{\oldsubparagraph{#1}\mbox{}}
\fi

%%% Use protect on footnotes to avoid problems with footnotes in titles
\let\rmarkdownfootnote\footnote%
\def\footnote{\protect\rmarkdownfootnote}

%%% Change title format to be more compact
\usepackage{titling}

% Create subtitle command for use in maketitle
\providecommand{\subtitle}[1]{
  \posttitle{
    \begin{center}\large#1\end{center}
    }
}

\setlength{\droptitle}{-2em}

  \title{Conservation Biology - Quantitative Methods Module}
    \pretitle{\vspace{\droptitle}\centering\huge}
  \posttitle{\par}
    \author{Yolanda Wiersma}
    \preauthor{\centering\large\emph}
  \postauthor{\par}
      \predate{\centering\large\emph}
  \postdate{\par}
    \date{2020-02-24}

\usepackage{booktabs}

\begin{document}
\maketitle

{
\setcounter{tocdepth}{1}
\tableofcontents
}
\chapter{Quantitative Methods
Overview}\label{quantitative-methods-overview}

Biology, including ecology and conservation, is becoming increasingly
quantitative. Familiarity with a range of statistical tools
(frequentist, multivariate, Bayesian) is expected. Many researchers draw
on quantitative methods; whether it is process or mathematical
modelling, programming, bioinformatics, or bioeconomic analyses. We will
only touch on key concepts in quantitative methods; in-depth training
would require multiple full-semester courses. Here, we will carry out a
few simple exercise that mimics some of the quantitative skills that you
need to be a successful ecologist. These are:

\begin{enumerate}
\def\labelenumi{\arabic{enumi}.}
\tightlist
\item
  Use R!
\item
  Manage data sets
\item
  Plot data
\item
  Fit a model to data
\end{enumerate}

In this module there will be a small assignment with each of the above
objectives. These are detailed below along with the percentage of your
final grade that this is worth (the entire module is worth 10\% of your
final mark).

\textbf{What to hand in (as a single PDF file please!)}

\begin{enumerate}
\def\labelenumi{\arabic{enumi}.}
\item
  Written review of the first-year guide. How useful was it to help you
  work through this module?
\item
  A short document that answers all the questions posed throughout this
  assignment (they are all shown in italics)
\item
  Copies of the final version of all plots created (remember you can
  export these as image files from the ``Plots'' pane in R Studio which
  you can then paste in Word). Convert the Word document to a PDF file
  and upload it to the Dropbox folder in Brightspace.
\end{enumerate}

\chapter{Using R}\label{using-r}

We're going to work exclusively with R in this module. Most of you have
some R/R studio experience. Dr.~Amy Hurford and I, together with
teaching assistant Joany Marino, have prepared a simple ``using R''
manual that we are planning to introduce to first year Biology students
in the near future. This manual is available by clicking
\href{https://ahurford.github.io/quantitative-training-guide/}{HERE}
(note it is still very much a DRAFT and thus has notes between the
authors in the document. Please ignore these).

\emph{Your assignment}: Provide a written review of the draft manual
that gives Dr.~Hurford and I some constructive suggestions on how to
improve this manual to make it as maximally useful for first year
students as possible. You do not need to highlight typos or any residual
``notes'' we've left each other, rather focus on a ``big picture''
evaluation of the document. This does not need to be long. It can be as
short as a paragraph/half-page of point form suggestions up to a maximum
of a full page typed-double spaces (apprx. 250 words).

\textbf{THIS IS WORTH 10\% OF YOUR MARK FOR THIS MODULE.}

\chapter{Manage a data set}\label{manage-a-data-set}

R is really useful for managing data. Here are some commands that come
in handy frequently. The reference to ``iris'' is the built-in ``iris''
data set that comes automatically when you install the baseR package.
Try out these commands by typing the code below into the console.
\emph{Write down what happens when you enter each command.}

\texttt{data("iris")}

\texttt{head(iris)}

\texttt{dim(iris)}

\texttt{str(iris)}

\texttt{class(iris)}

Let's say you want to create a new data frame from the ``iris'' data
frame, but with only the data for flowers with a petal width greater
than 0.3. Here's what the command would look like:

\texttt{lg.petals\ \textless{}-\ subset(iris,\ Petal.Width\ \textgreater{}\ 0.3)}

\emph{How can you confirm this new data frame is a subset of the larger
one?}

\emph{Your assignmnet} Create new data frames from the ``iris'' data set
that contain the following. \emph{Provide a copy of the code that you
used to generate each new data frame and tell me how many rows are in
each new data frame.} If you get stuck, start with the first year guide!
Then try the ``help'' files (the guide includes a section on getting
help).

\begin{enumerate}
\def\labelenumi{\arabic{enumi}.}
\item
  A data frame with only flowers with Sepal Length less than 4.5
\item
  A data frame with only flowers with Sepal Width greater than 3.3
\item
  A data frame with only the species ``setosa'' (Warning - the
  ``species'' field is factor data, which means you need to treat it a
  little differently! Use the help files, or Google it!)
\item
  \textbf{For graduate students (optional for keen undergraduates}):
  Create a data frame of all species ``setosa'' that have Petal.Width
  great than 0.3. \emph{Tell me how big this data frame is.}
\end{enumerate}

\textbf{THIS IS WORTH 25\% OF YOUR MARK FOR THIS MODULE.}

\chapter{Plotting Data}\label{plotting-data}

The first year help file has a section on how to plot data when you have
x and y vectors. Use this to make a plot of the iris data that shows
Sepal Length vs.~Sepal Width. I've provided code to create the x and y
vectors below. Remember to add axis labels!

\texttt{x\ \textless{}-\ iris\$Sepal.Width}

\texttt{y\ \textless{}-\ iris\$Sepal.Length}

\emph{Your assignment} Modify the graph you created above to show the
three species with different symbols. There are different ways you can
do this (\textbf{HINT} look at the goldfish data example in the
first-year help manual).

As an alternative to the way it's done in the goldfish example in the
first-year manual, you can use the \texttt{pch} (for symbol style) and
\texttt{bg} (for background colour) arguments to specify three colours
for the species. Here's a snippet of the code that will assign the
colours red, green and blue to the 3 factors in the data set:
\texttt{pch\ =\ 21,\ bg\ =\ c("red",\ "green",\ "blue")} \textbf{HINT}
insert this code into the \texttt{plot} function.

To make a legend use this code:
\texttt{legend(2,\ 7.5,\ pch=\ 16,\ leg.text,\ col\ =\ c("red",\ "green",\ "blue"))}.
Experiment with changing the variables in this line of code to see what
happens!

\emph{For graduate students (optional for keen undergraduates}): Make a
boxplot of sepal lengths by species. \textbf{HINT} you need to insert a
formula of the form \texttt{y\textasciitilde{}grp} where \texttt{y} is a
numerical vector of the data you want split into the boxplots and
\texttt{grp} is the groupings you want to show on the x-axis (i.e., the
factor data). Don't forget meaningful axis labels!

\textbf{THIS IS WORTH 30\% OF YOUR MARK FOR THIS MODULE.}

\chapter{Fitting a model}\label{fitting-a-model}

Let's say you want to test whether there is a statical relationship
between sepal length and sepal width in the ``Versicolor'' Species of
iris. We suspect it is a linear fit, so we'll just fit a linear model of
the form y = a(x) + b, where ``y'' is sepal length, x is sepal width, a
is the coefficient for the slope of the line, and b is the intercept.

We'll use the glm (generalized linear model) not the ls (least squares)
model, since it allows for better post-hoc model evalution. To do that
we need to load and install the package ``vegan''.

\emph{Your assignment}

\begin{enumerate}
\def\labelenumi{\arabic{enumi}.}
\item
  Install the package ``vegan''. **HINT* Refer to the instructions on
  using packages in the first-year R guide.
\item
  Subset the data frame ``iris'' to create a new data frame that only
  has the ``Versicolor'' species (\textbf{HINT} you already did
  somehting like this, but for a different species in step 3 of the
  ``Managing a data set'' part of the assignment).
\item
  \emph{Inspect new data frame and write down how many rows are in it.}
\item
  Fit a model by writing code that tells R to create an object called
  \texttt{Sepal.Model} using the function \texttt{glm}. Write the code
  as below, but where I have things inside \textless{}\textgreater{}
  replace with correct parameters from your new data frame. \emph{Give a
  copy of the code when you hand in the work}
\end{enumerate}

\texttt{Sepal.Model\ \textless{}-\ glm(\textless{}y\textgreater{}\ \textasciitilde{}\ \textless{}x\textgreater{},\ data\ =\ \textless{}new\ data\ frame\textgreater{})}

If you execute this code and nothing happens, then it has worked. You
can see a summary of the model fit with the command
\texttt{summary(Sepal.Model)}

\emph{Is there a significant relationship between Sepal Length and Sepal
width? What's the coefficeint?}

It's always important to check that your model doesn't violate
assmptions of normality. You can assess this by inspecting the ratio of
the null deviance to residual deviance in the model summary.

You can also inspect the model by plotting the model fits. Do this with
the code \texttt{plot(Speal.Model)}. You will see plots of 1) residuals
vs.~fitted data; 2) a Q-Q plot; 3) Scale-location plot; 4) Residulas
vs.~Leverage.

Click the links here to read about how to interpret the
\href{https://stats.stackexchange.com/questions/76226/interpreting-the-residuals-vs-fitted-values-plot-for-verifying-the-assumptions}{RESIDUALS
vs FITTED PLOT}; the
\href{https://stats.stackexchange.com/questions/101274/how-to-interpret-a-qq-plot}{Q-Q
PLOT\textgreater{}}. If your model does not meet the assumptions of
normality, you may need to use a different error strucutre (e.g.,
Poisson, logNormal). Going into the whys and hows of that is beyond the
scope of this course. The fit here is slightly skewed but the null and
residual deviances are small enought that we can accept this fit.

Of course, you probably want to plot a line of best fit! To do that, we
create a range of values for ``x'' (Sepal Width) that matches the range
of the actual data. It's helpful to plot the data first.

So\ldots{} make a scatter plot with ``Sepal Width'' on the x-axis and
``Sepal Length'' on the y-axis. \textbf{HINT} refer back to the
``Plotting Data'' part of the module.

Note, you could also query the data to see what the range of real x
values are by typing \texttt{range(iris\$Sepal.Width)}.

Then specify a range of values that fall within this for fitting the
model. We'll create a sequence called ``xvalues'' that ranges between 2
and 5 in increments of 0.1 using this code
\texttt{xvalues\ \textless{}-\ seq(2,5,\ 0.1)}. \textbf{Aside} you've
probably been ignoring the other windows in R studio, but notice how the
``xvalues'' popped up in the ``Environment'' window and shows you the
length of the vector and the first values.

Then we use the \texttt{predict} function to create the model for these
``xvalues'' we just created, based on the ``Sepal.model'' you built with
the glm.

\texttt{yvalues\ \textless{}-\ predict(Sepal.Model,\ list(Sepal.Width\ =\ xvalues),\ type="response")}

Then simply add the line of fit to your plot with
\texttt{lines(xvalues,\ yvalues).}

\textbf{THIS IS WORTH 35\% OF YOUR MARK FOR THIS MODULE.}

\chapter{Other sources of info}\label{other-sources-of-info}

R ``cheat sheets'' can be really handy. See this
\href{https://rstudio.com/wp-content/uploads/2016/05/base-r.pdf}{LINK}
for an example cheat sheet for baseR. There are other cheat sheet for
other R packages. For example, I wrote this document using Rmarkdown
which has a cheat sheet
\href{https://rstudio.com/wp-content/uploads/2015/02/rmarkdown-cheatsheet.pdf}{HERE}.

If you want more detailed help using R than what's in the first year
guide provided as part of this module, check out ``The Student's Guide
to R'' by Nochlas Horton, Randall Pruim and Daniel Kaplan. It's
availalbe online
\href{https://cran.r-project.org/doc/contrib/Horton+Pruim+Kaplan_MOSAIC-StudentGuide.pdf}{HERE}

\bibliography{book.bib,packages.bib}


\end{document}
